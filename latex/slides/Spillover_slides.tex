\documentclass[aspectratio=169]{beamer}


% Title ------------------------------------------------------------------------------
\title{Difference-in-Differences with Spatial Spillovers}
\date{\today}
\author{Kyle Butts}
% \institue{}

% xcolor and define colors -------------------------
\usepackage{xcolor}
% https://www.materialpalette.com/colors
\definecolor{red}{HTML}{c62828}
\definecolor{orange}{HTML}{ef6c00}
\definecolor{green}{HTML}{2e7d32}
\definecolor{blue}{HTML}{1565c0}
\definecolor{purple}{HTML}{283593}
\definecolor{maroon}{HTML}{AF3335}
\definecolor{dark-maroon}{HTML}{5D0F0D}
\definecolor{teal}{HTML}{00695c}
\definecolor{bluegrey}{HTML}{455a64}

\definecolor{buff-gold}{HTML}{CFB87C}
\definecolor{buff-grey}{HTML}{565A5C}
\definecolor{buff-lightgrey}{HTML}{A2A4A3}
\definecolor{buff-black}{HTML}{000000}


% Enumerate/Itemize --------------------------------
\usepackage{enumitem}
\setlist[itemize]{label= \textbullet}

% Hyperlinks ---------------------------------------
\usepackage{hyperref}
\hypersetup{
    colorlinks= true,
    citecolor= dark-maroon,
    linkcolor= dark-maroon,
    filecolor= dark-maroon,      
    urlcolor= dark-maroon,
}

% AMS ----------------------------------------------
\usepackage{amsmath}
\usepackage{amsfonts}

% AMS Theorem --------------------------------------
\usepackage{amsthm}

% note, theorem is the name that goes in \begin{} and Theorem is the name displayed as Theorem 1
\newtheorem{theorem}{Theorem}
\newtheorem{assumption}{Assumption}

% Font ---------------------------------------------
\usepackage[libertine]{newtxmath}


% Section and Subsection Styling -------------------
\usepackage{titlesec}

\titleformat{\section}
  {\Large \bf \color{teal}}
  {\thesection.}
  {1em}
  {}
  
\titleformat{\subsection}
  {\large \it \color{teal}}
  {\thesubsection}
  {1em}
  {}

% Better Abstract ----------------------------------
\renewenvironment{abstract}
{
  \centerline
  {\large \bfseries \color{teal} Abstract}
  \begin{quote}
}
{
  \end{quote}
}

% Margins ------------------------------------------
\usepackage[margin=1.5in]{geometry}

% Line Spacing -------------------------------------
\renewcommand{\baselinestretch}{1.5}

% Remove annoying over-full box --------------------
\vfuzz2pt 
\hfuzz2pt

% Packages for Better Tables ------------------------
\usepackage{booktabs}
\usepackage{longtable}
\usepackage{lscape}



% --------------------------------------------------

\addbibresource{references.bib}

% ------------------------------------------------------------------------------
\begin{document}

% ------------------------------------------------------------------------------
\maketitle
% ------------------------------------------------------------------------------

\begin{frame}{Spatial Spillovers}
    Researchers aim to estimate the \textbf{average treatment effect on the treated}: 
    \[
        \tau \equiv \mathbb{E} \left[ Y_{i1}(1) - Y_{i1}(0) \ \vert \ D_{i} = 1 \right]
    \]
    
    Estimation is complicated by \textbf{Spillover Effects}
    
    \vspace{5mm}
    \textbf{Spillover effects} occur when the effect of treatment extends over the treatment boundaries (states, counties, etc.). Example: Amazon Shipping Center
    
    \begin{itemize}
        \item A shipping center opening in county \textit{c} has positive employment effects on \textbf{nearby control counties}
        
        \item Having nearby counties with shipping centers raises employment and therefore reduces the effect of \textbf{treated counties} through general equilibrium effects
    \end{itemize}
\end{frame}

\begin{frame}{This Paper}
    In this paper, I...

    \begin{itemize}
        \item Present a potential outcomes framework to formalize spillover effects and evaluate ad-hoc adjustments commonly found in the literature
        
        \item Propose an estimation strategy that improves on current practices by being more robust to spillovers
        
        \item Apply this framework to improve estimation of the local effect of place-based policies in Urban Economics 
    \end{itemize}

\end{frame}


\begin{frame}{Bias from Spatial Spillovers}
    
    \onslide<1->{
        The canonical difference-in-differences estimate is: 
        \only<1>{
            \[ 
                \hat{\tau} = \underbrace{\hat{\mathbb{E}} \left[ Y_{i1} - Y_{i0} \mid D_i = 1 \right]}_{\text{Counterfactual Trend} \ + \ \tau} - 
                \underbrace{\hat{\mathbb{E}} \left[ Y_{i1} - Y_{i0} \mid D_i = 0 \right]}_{\text{Counterfactual Trend}}
            \]
        }
        \only<2>{
            \[ 
                \hat{\tau} = \underbrace{\hat{\mathbb{E}} \left[ Y_{i1} - Y_{i0} \mid D_i = 1 \right]}_{\text{Counterfactual Trend} \ + \ \tau} - 
                \underbrace{\hat{\mathbb{E}} \left[ Y_{i1} - Y_{i0} \mid D_i = 0 \right]}_{\substack{\text{Counterfactual Trend} \\[2mm] \ + \ \text{\color{coral} Spillover on Control}}}
            \]
        }
        \only<3>{
            \[ 
                \hat{\tau} = \underbrace{\hat{\mathbb{E}} \left[ Y_{i1} - Y_{i0} \mid D_i = 1 \right]}_{\substack{\text{Counterfactual Trend} \ + \ \tau \\[2mm] \ + \ \text{\color{kelly} Spillover on Treated}}} - 
                \underbrace{\hat{\mathbb{E}} \left[ Y_{i1} - Y_{i0} \mid D_i = 0 \right]}_{\substack{\text{Counterfactual Trend} \\[2mm] \ + \ \text{\color{coral} Spillover on Control}}}
            \]
        }
    } 

    Two problems occur in the presence of spillover effects:
    
    
    \begin{itemize}
        \onslide<2->{
            \item {\bf \color{coral} Spillover onto Control Units:} 
            
            Nearby ``control'' units fail to estimate counterfactual trends because they are affected by treatment
        }
        
        \onslide<3->{
            \vspace{2.5mm}
            \item {\bf \color{kelly} Spillover onto other Treated Units:} 
            
            Treated units are also affected by nearby units and therefore combine ``direct'' effects with spillover effects
        }
    \end{itemize}

\end{frame}

\begin{frame}{Remove Bias}
    \only<1>{
        \[ y_{it} = \mu_t + \mu_i + \tau D_{it} + \varepsilon_{it} \]
    }
    \only<2>{
        \[ 
            y_{it} = \mu_t + \mu_i + \tau D_{it} + {\color{coral} \tau_{\text{spill,control}} \text{Within}_{it} * (1-D_{it}) } + {\color{kelly} \tau_{\text{spill,treat}} \text{Within}_{it} * D_{it}} +  + \varepsilon_{it}
        \]
    }

    \only<1>{ 
        \[ 
            \mathbb{E}[\hat{\tau}] = \tau + \text{\color{coral} Spillover on Control} + \text{\color{kelly}Spillover on Treated}
        \]
    }
    \only<2>{
        \[ 
            \mathbb{E}[\hat{\tau}] = \tau,
        \]
        so long as $\text{Within}_{it}$ contains all the units with spillovers.
    }
\end{frame}


\begin{frame}
    \begin{figure}[tb!]
        \caption{Single Ring - Removes spillover effects}
        \begin{center}
            \resizebox{0.9\textwidth}{!}{
                \includegraphics{../../figures/figure-within_example.pdf}
            }
        \end{center}
    \end{figure}
\end{frame}

\begin{frame}
    \begin{figure}[tb!]
        \caption{Multiple Rings - Improves estimation of spillover effects}
        \begin{center}
            \resizebox{0.9\textwidth}{!}{
                \includegraphics{../../figures/figure-rings_example.pdf}
                } 
            \end{center}
    \end{figure}
\end{frame}

% \begin{frame}
%     \begin{figure}[tb!]
%         \caption{Community Health Centers}
%         \begin{center}
%             \resizebox{0.9\textwidth}{!}{
%                 \includegraphics{../../figures/figure-chc-es_combined_slides.pdf}
%                 } 
%             \end{center}
%     \end{figure}
% \end{frame}


% Grey out overlays
\setbeamercovered{transparent}

\begin{frame}{Outline}

    \begin{itemize}
        \item[1--] Formalize spillovers into a potential outcomes framework:
 
        \begin{citecolor}
            [\citet{Clarke_2017}, \citet{Berg_Streitz_2019}, and \citet{Verbitsky-Savitz_Raudenbush_2012}]
        \end{citecolor}

        \begin{itemize}
            \vspace{2.5mm}
            \item I decompose the difference-in-differences estimator into three parts: Direct Effect of Treatment, Spillover onto Treated Units, Spillover onto Control Units
            
            \vspace{2.5mm}
            \item Show that an indicator for being close to treated units remove \textit{all bias} so long as the indicator contains all units affected by spillovers
            
            \vspace{2.5mm}
            \item `Rings' are able to estimate spillover effects while still removing all bias
        \end{itemize} 

    \end{itemize}
\end{frame}

\begin{frame}{Outline}
    \begin{itemize}
        \item[2--] Apply framework to Urban Economics
        \begin{itemize}
            \item Revisit \begin{citecolor}\citet{Kline_Moretti_2014a}\end{citecolor} analysis of the Tennessee Valley Authority
            
            \begin{itemize}
                \vspace{2.5mm}
                \item The local effect estimate is contaminated by spillover effects to neighboring counties \begin{citecolor}\citep{Kline_Moretti_2014b}\end{citecolor}
                
                \vspace{2.5mm}
                \item Large scale manufacturing investment creates an `urban shadow' \begin{citecolor}\citep{Cuberes_Desmet_Rappaport_2021,Fujita_Krugman_Venables_2001}\end{citecolor}
            \end{itemize}
            
            \vspace{2.5mm}
            \item Discuss how framework can reconcile conflicting findings on effect of federal Empowerment Zones \begin{citecolor}\citep{Busso_Gregory_Kline_2013,Neumark_Kolko_2010}\end{citecolor}
            
            \vspace{2.5mm}
            \item Event Study estimates of Community Health Centers find highly localized efffects \begin{citecolor}\citep{Bailey_Goodman_Bacon_2015}\end{citecolor}
        \end{itemize}
    \end{itemize} 
\end{frame}



% ------------------------------------------------------------------------------
\section{Theory}
% ------------------------------------------------------------------------------


\begin{frame}{Potential Outcomes Framework}
    For exposition, I will label units as counties. Assume all treatment occurs at the same time (2-periods or pre-post averages).\footnote{I extend this into an event study framework in the paper, but the intuition is the same as in the $2 \times 2$ setting.}
    
    \begin{itemize}
        \item $Y_{it}(D_i, \textcolor{alice}{h(\vec{D}, i)})$ is the potential outcome of county $i \in \{ 1, \dots, N \}$ at time $t$ with treatment status $D_i \in \{0, 1\}$.
        
        \item $\vec{D} \in \{0,1\}^N$ is the vector of all units treatments.
        
        \pause
        \item The function $\textcolor{alice}{h(\vec{D}, i)}$ maps the entire treatment vector into an `exposure mapping' which can be a scalar or a vector.
        
        \pause
        \item No exposure is when $\textcolor{alice}{h(\vec{D}, i)} = \vec{0}$.
    \end{itemize}
\end{frame}

% Grey out
% \setbeamercovered{transparent}

\begin{frame}{Examples of $h_i(\vec{D})$}
    
    Examples of $h_i(\vec{D})$:
    
    \begin{itemize}
        \item \textbf{Treatment within $x$ miles:}
        
        $\textcolor{alice}{h(\vec{D}, i)} = max_j \ 1\left(d(i, j\right) \leq x)$ where $d(i,j)$ is the distance between counties $i$ and $j$. 

        \begin{itemize}
            \item e.g. library access where $x$ is the maximum distance people will travel
            
            \item Spillovers are non-additive, i.e. spillover effects do not depend on number of nearby treated areas
        \end{itemize}

    \end{itemize}
\end{frame}

\imageframe{../../figures/figure-spill_within.png}

\begin{frame}{Examples of $h_i(\vec{D})$}
    
    Examples of $h_i(\vec{D})$:
    
    \begin{itemize}
        \item \textbf{Treatment within $x$ miles:}
        
        $\textcolor{alice}{h(\vec{D}, i)} = max_j \ 1(d(i, j) \leq x)$ where $d(i,j)$ is the distance between counties $i$ and $j$. 

        \begin{itemize}
            \item e.g. library access where $x$ is the maximum distance people will travel
            
            \item Spillovers are non-additive
        \end{itemize}
        
        \vspace{2.5mm}
        \item \textbf{Number of Treated within $x$ miles:}
        
        $\textcolor{alice}{h(\vec{D}, i)} = \sum_{j = 1}^k 1(d(i, j) \leq x)$. 

        \begin{itemize}
            \item e.g. Amazon shipping center
            
            \item Agglomeration economies suggest spillovers are additive
        \end{itemize}

    \end{itemize}
\end{frame}


\imageframe{../../figures/figure-spill_within_additive.png}


\begin{frame}{Estimand of Interest}
    \begin{block}{Estimand of Interest:}
    \[ 
        \textcolor{green}{\tau_{\text{direct}}} \equiv \textcolor{green}{\mathbb{E}\left[ Y_{i,1}(1, 0) - Y_{i,1}(0, 0) \mid D_i = 1\right]}
    \]
    \end{block}


    \pause
    \vspace{2.5mm}
    \begin{block}{Spillover Effects:}
    \[
        \textcolor{red}{\tau_{\text{spillover, treated}}} \equiv \textcolor{red}{\mathbb{E} \left[ Y_{i1}(1, h_i(\vec{D})) - Y_{i1}(1, 0) \mid D_i = 1 \right]}
    \]

    \[ 
        \textcolor{coral}{\tau_{\text{spillover, control}}} = \textcolor{coral}{\mathbb{E} \left[ Y_{i1}(0, h_i(\vec{D})) - Y_{i1}(0, 0) \mid D_i = 0 \right]}
    \]
    \end{block}
\end{frame}


\begin{frame}{Parallel Trends}
    I assume a modified version of the parallel counterfactual trends assumption: 

    \vspace{2.5mm}
    \begin{block}{Assumption: {\it Parallel Counterfactual Trends}}
    \begin{align*}
        \mathbb{E}\big[ \underbrace{Y_{i,1}(0, \textcolor{alice}{\vec{0}}) - Y_{i,0}(0, \textcolor{alice}{\vec{0}})}_{\text{Counterfactual Trend}} \mid D_i = 1 \big] = \ \mathbb{E}\big[ \underbrace{Y_{i,1}(0, \textcolor{alice}{\vec{0}}) - Y_{i,0}(0, \textcolor{alice}{\vec{0}})}_{\text{Counterfactual Trend}} \mid D_i = 0 \big]
    \end{align*}
    \end{block}

    \vspace{5mm}
    In the \textit{complete absence of treatment }(not just the absence of individual $i$'s treatment):
    
    Changes in outcomes do not depend on treatment status
    
\end{frame}

\begin{frame}{What does Difference-in-Differences identify?}
    With the parallel trends assumption and random assignment of $D_i$, I decompose the difference-in-differences estimate as follows: 
        
    \begin{align*}
        \mathbb{E}\left[\hat{\tau}\right] &= \underbrace{\mathbb{E} \left[ Y_{i1} - Y_{i0} \mid D_i = 1 \right] - \mathbb{E} \left[ Y_{i1} - Y_{i0} \mid D_i = 0 \right]}_{\text{Difference-in-Differences}} \\[3mm]
        \pause&= 
        \textcolor{green}{\mathbb{E} \left[ Y_{i1}(1, 0) - Y_{i1}(0, 0) \mid D_i = 1 \right]} \\
        &\quad + \quad 
        \textcolor{red}{\mathbb{E} \left[ Y_{i1}(1, h_i(\vec{D})) - Y_{i1}(1, 0) \mid D_i = 1 \right]} \\ 
        &\quad - \quad  
        \textcolor{coral}{\mathbb{E} \left[ Y_{i1}(0, h_i(\vec{D})) - Y_{i1}(0, 0) \mid D_i = 0 \right]} \\[3mm]
        &= \textcolor{green}{\tau_{\text{direct}}} + \textcolor{red}{\tau_{\text{spillover, treated}}} - \textcolor{coral}{\tau_{\text{spillover, control}}}
    \end{align*}

\end{frame}


\begin{frame}{Removing Bias}
    
    \begin{block}{Assumption: {\it Spillovers are Local}}

    Let $d(i,j)$ be the distance between units $i$ and $j$. There exists a distance $\bar{d}$ such that 
        
        (i) For all units $i$,
        \[ 
            \min_{j: D_j = 1} d(i,j) > \bar{d} \implies h(\vec{D}, i) = \vec{0}. 
        \]
    
        (ii) There are treated and control units such that $\min_{j: \ D_j = 1} d(i,j) > \bar{d}$.

    \end{block}
\end{frame}

\begin{frame}{Removing Bias}
    With assumption that spillovers are local, define $S_it$ to be an indicator equal to one in the post period for all units with $h(\vec{D}, i) \neq \vec{0}$ (and potentially some units with $= \vec{0}$).
    
    Estimation of the following equation:
    \[ 
        y_{it} = \mu_t + \mu_i + \tau D_{it} + \tau_{\text{spill,treat}} S_{it} * D_{it} + \tau_{\text{spill,control}} S_{it} * (1-D_{it}) + \varepsilon_{it}
    \]

    \begin{itemize}
        \item $\hat{\tau}$ is consistent for $\tau_{\text{direct}}$
        
        \item $\hat{\tau}_{\text{spill}}$'s are not consistent for average spillover effects (hence the need for rings).
    \end{itemize}

\end{frame}

\begin{frame}{Aside -- Removing ``contaminated'' controls}
    A common solution to the problem of spillover is to re-estimate on a subsample with neighboring control units removed.

    This is not recommended for two reasons:

    \begin{itemize}
        \item Removing control units from the sample decreases precision of the estimates
        
        \item Spillover effects on treated units will still remain in the estimand. 
    \end{itemize}
\end{frame}


% ------------------------------------------------------------------------------
\section{Estimation of Spillovers}
% ------------------------------------------------------------------------------


\begin{frame}{Robustness to Misspecification}
    Generate data using the following data-generating process with different exposure mappings:

    \[ 
        y_{\it} = \mu_t + \mu_i + 2 D_{it} + \beta_{\text{spill,control}} * (1-D_{it}) h(\vec{D}, i) + \varepsilon_{it}
    \]

    \begin{itemize}
        \item Observation $i$ is a U.S. county, year $t \in \{2000, \dots, 2019\}$, treatment $D_{it}$ turns on in 2010 and is assigned randomly.
        
        \item Details in the paper, but $\mu_t$ is normally distributed on a linear time trend, $\mu_i$ are normally distributed
    \end{itemize}

    Then, I estimate each data-generating process using (potentially) misspecified $\tilde{h}(\vec{D}, i)$ and report the average estimate bias.
\end{frame}

\begin{frame}{Specifications of $h(\vec{D}, i)$}
    
    \begin{itemize}
        \item \textit{Within $40/80$mi.}:
        \begin{itemize}
            \item Indicator for nearest treated unit being within $40/80$ miles.
        \end{itemize}
        
        \item \textit{Within $40/80$mi. (Additive)}: 
        \begin{itemize}
            \item Number of treated units being within $40/80$ miles.
        \end{itemize}
        
    \end{itemize}

\end{frame}

\imageframe{../../figures/figure-spill_within_large.png}
\imageframe{../../figures/figure-spill_within_large_additive.png}

\begin{frame}{Specifications of $h(\vec{D}, i)$}
    
    \begin{itemize}
        \item \textit{Decay}: 
        \begin{itemize}
            \item $\max_j D_j * e^{-0.02 d(i,j)} * 1(d(i,j) < 80)$
        \end{itemize}
        
        \item \textit{Decay (Additive)}: 
        \begin{itemize}
            \item $\sum_j D_j * e^{-0.02 d(i,j)} * 1(d(i,j) < 80)$
        \end{itemize}
        
    \end{itemize}

\end{frame}


\imageframe{../../figures/figure-spill_decay.png}
\imageframe{../../figures/figure-spill_decay_additive.png}


\begin{frame}{Specifications of $h(\vec{D}, i)$}
    
    \begin{itemize}
        \item \textit{Rings}: 
        \begin{itemize}
            \item Set of concentric rings. For each ring, indicator for nearest treated unit being within that distance bin
        \end{itemize}
        
        \item \textit{Rings (Additive)}: 
        \begin{itemize}
            \item Set of concentric rings. For each ring, number of treated units being within that distance bin
        \end{itemize}
    \end{itemize}

\end{frame}

\imageframe{../../figures/figure-spill_ring.png}

\begin{frame}

    \begin{table}[!tb]
        \caption{Bias from Misspecification of Spillovers}
        \label{tab:misspecification}
    
        \begin{adjustbox}{width = \textwidth, center}
            \begin{threeparttable}
                \begin{tabular}{@{} l rrrrrr @{}}
                    % Head
                    \toprule
                    & \multicolumn{6}{c}{Data-Generating Process} \\
                    \cmidrule{2-7}
    
                    & \multicolumn{1}{c}{\textbf{Within 40mi.}} & \multicolumn{1}{c}{\textbf{Within 80mi.}} & \multicolumn{1}{c}{\textbf{Within 40mi.}} & \multicolumn{1}{c}{\textbf{Within 80mi.}} & \multicolumn{1}{c}{\textbf{Decay 80mi.}} & \multicolumn{1}{c}{\textbf{Decay 80mi.}} \\
                    Specification & & & \multicolumn{1}{c}{\textbf{(Additive)}} & \multicolumn{1}{c}{\textbf{(Additive)}} & & \multicolumn{1}{c}{\textbf{(Additive)}} \\
    
     
                    % Body
                    \midrule
                    
                    
                    % \input{../../tables/misspecification_slides.tex}
                    TWFE (No Spillovers) & $0.258$ & $0.258$ & $0.258$ & $0.258$ & $0.258$ & $0.258$ \\ \midrule
                    \marktopleft{c1}Within 40mi. & $-0.005$ & $0.213$ & $-0.005$ & $0.176$ & $0.159$ & $0.143$ \\ 
                    Within 80mi. & $-0.009$ & $-0.009$ & $-0.009$ & $-0.009$ & $-0.009$ & $-0.009$ \markbottomright{c1}
                    \onslide<2->{
                        \\ \midrule
                        Within 40mi. (Additive) & $0.043$ & $0.221$ & $-0.006$ & $0.177$ & $0.174$ & $0.143$ \\ 
                        Within 80mi. (Additive) & $0.034$ & $0.134$ & $-0.012$ & $-0.009$ & $0.099$ & $-0.010$ \\ \midrule
                        Decay 80mi. & $-0.159$ & $0.070$ & $-0.174$ & $0.014$ & $-0.009$ & $-0.033$ \\ 
                        Decay 80mi. (Additive) & $-0.023$ & $0.148$ & $-0.084$ & $0.019$ & $0.088$ & $-0.008$
                    }
                    \onslide<3->{
                        \\ \midrule
                        \marktopleft{c2}Rings (0-20, 20-30, 30-40) & $-0.005$ & $0.213$ & $-0.005$ & $0.176$ & $0.159$ & $0.143$ \\ 
                        Rings (0-20, 20-30, 30-40, 40-60, 60-80) & $-0.009$ & $-0.009$ & $-0.009$ & $-0.009$ & $-0.009$ & $-0.009$ \markbottomright{c2}\\ 
                        Rings (0-20, 20-30, 30-40, 40-60, 60-80) (Additive) & $0.036$ & $0.134$ & $-0.008$ & $-0.008$ & $0.100$ & $-0.009$ 
                    }
                        
                    \\ \bottomrule
                \end{tabular}
            \end{threeparttable}
        \end{adjustbox}
    \end{table}

    \onslide<3->{
        \textbf{Indicator (or set of rings) that captures \emph{all} affected unit removes all bias}
    }
\end{frame}

\begin{frame}{Spillovers as Estimand of Interest}
    Until now, we assumed our estimand of interest is $\textcolor{green}{\tau_{\text{direct}}}$.
    
    However, the two other spillover effects are of interest as well:
    \begin{itemize}
        \item $\textcolor{coral}{\tau_{\text{spillover, control}}}$: Do the benefits of a treated county come at a cost to neighbor counties? 
        
        \item $\textcolor{red}{\tau_{\text{spillover, treated}}}$: Does the estimated effect change based on treatment of neighbors? 
    \end{itemize}
    
    To estimate the spillover effects, we have to parameterize $h(\vec{D}, i)$ function and the potential outcomes function $Y_i(D_i, h(\vec{D}, i))$.
\end{frame}

\begin{frame}{Estimation of Spillover Effects}
    To see which specifications can predict spillover effects well, I estimate the spillover effects for each control unit, $\hat{\beta}_{\text{spill, control}} * \tilde{h}(\vec{D}, i)$.

    Then calculate \[ 
        1 - \frac{
                \overbrace{\sum_{i: D_i = 0} (\beta_{\text{spill, control}} h(\vec{D}, i) - \hat{\beta}_{\text{spill, control}} \tilde{h}(\vec{D}, i))^2}^{\text{Mean Square Prediction Error}}
            }{
                \underbrace{\sum_{i: D_i = 0} (\beta_{\text{spill, control}} h(\vec{D}, i))^2}_{\text{Normalization}}
            }    
    \]

    This gives the proportion of spillovers explained by $\tilde{h}(\vec{D}, i)$

\end{frame}

\begin{frame}
    \begin{figure}[tb!]
        \caption{Single Ring - Removes spillover effects}
        \begin{center}
            \resizebox{0.9\textwidth}{!}{
                \includegraphics{../../figures/figure-within_example.pdf}
            }
        \end{center}
    \end{figure}
\end{frame}

\begin{frame}
    \begin{figure}[tb!]
        \caption{Multiple Rings - Improves estimation of spillover effects}
        \begin{center}
            \resizebox{0.9\textwidth}{!}{
                \includegraphics{../../figures/figure-rings_example.pdf}
                } 
            \end{center}
    \end{figure}
\end{frame}


\begin{frame}
    \begin{table}[!tb]
        \caption{Percent of Spillovers Predicted by Specification}
        \label{tab:misspecification_mspe}
    
        \begin{adjustbox}{width = \textwidth, center}
            \begin{threeparttable}
                \begin{tabular}{@{} l rrrrrr @{}}
                    % Head
                    \toprule
                    & \multicolumn{6}{c}{Data-Generating Process} \\
                    \cmidrule{2-7}
    
                    & \multicolumn{1}{c}{\textbf{Within 40mi.}} & \multicolumn{1}{c}{\textbf{Within 80mi.}} & \multicolumn{1}{c}{\textbf{Within 40mi.}} & \multicolumn{1}{c}{\textbf{Within 80mi.}} & \multicolumn{1}{c}{\textbf{Decay 80mi.}} & \multicolumn{1}{c}{\textbf{Decay 80mi.}} \\
                    Specification & & & \multicolumn{1}{c}{\textbf{(Additive)}} & \multicolumn{1}{c}{\textbf{(Additive)}} & & \multicolumn{1}{c}{\textbf{(Additive)}} \\
    
     
                    % Body
                    \midrule
                    
                    
                    % \input{../../tables/misspecification_slides.tex}
                    TWFE (No Spillovers) & $0.0\%$ & $0.0\%$ & $0.0\%$ & $0.0\%$ & $0.0\%$ & $0.0\%$ \\ \midrule
                    \marktopleft{c3}Within 40mi. & $99.4\%$ & $25.9\%$ & $85.6\%$ & $38.8\%$ & $59.5\%$ & $56.1\%$ \\ 
                    Within 80mi. & $39.8\%$ & $96.2\%$ & $34.3\%$ & $71.7\%$ & $85.6\%$ & $68.0\%$ \markbottomright{c3}
                    \onslide<2->{
                    \\ \midrule
                        Within 40mi. (Additive) & $85.3\%$ & $21.2\%$ & $99.5\%$ & $40.6\%$ & $52.0\%$ & $60.7\%$ \\ 
                        Within 80mi. (Additive) & $45.8\%$ & $61.8\%$ & $47.2\%$ & $98.4\%$ & $71.0\%$ & $93.6\%$ \\ \midrule 
                        Decay 80mi. & $60.1\%$ & $82.5\%$ & $52.7\%$ & $75.8\%$ & $97.5\%$ & $82.2\%$ \\ 
                        Decay 80mi. (Additive) & $60.7\%$ & $56.9\%$ & $63.8\%$ & $93.5\%$ & $79.0\%$ & $98.7\%$ 
                    }
                    \onslide<3->{
                        \\ \midrule 
                        \marktopleft{c4}Rings (0-20, 20-30, 30-40) & $98.4\%$ & $23.7\%$ & $85.9\%$ & $37.5\%$ & $58.9\%$ & $56.2\%$ \\ 
                        Rings (0-20, 20-30, 30-40, 40-60, 60-80) & $96.6\%$ & $91.7\%$ & $84.2\%$ & $72.7\%$ & $91.9\%$ & $78.4\%$ \\ 
                        Rings (0-20, 20-30, 30-40, 40-60, 60-80) (Additive) & $83.5\%$ & $57.4\%$ & $97.6\%$ & $95.0\%$ & $73.5\%$ & $94.9\%$\markbottomright{c4} 
                    }
                        
                    
                    \\ \bottomrule
                \end{tabular}
            \end{threeparttable}
        \end{adjustbox}
    \end{table}

    \onslide<3->{
        \textbf{Donuts perform best at estimating spillover effects.}
    
        \textbf{It is important to get Additive vs. Non-Additive correct.}
    }
\end{frame}




% ------------------------------------------------------------------------------
\section{Application in Urban Economics}
% ------------------------------------------------------------------------------

\begin{frame}{Tennessee Valley Authority}
    \citet{Kline_Moretti_2014a} look at the long-run impacts of the Tennessee Valley Authority (TVA).

    \begin{itemize}
        \item The TVA was a large-scale federal investment started in 1934 that focused on improving manufacturing economy. (Hundreds of dollars spent anually per person)
        
        \item The program focused on large-scale dam construction that brought cheap wholesale electricity to the region
    \end{itemize}

    \pause
    Research Questions:
    \begin{itemize}
        \item What are the local effects of TVA on manufacturing and agricultural economies?
        \item Do these effects come at the cost of other counties?
    \end{itemize}
\end{frame}

\begin{frame}{Identification}
    \citet{Kline_Moretti_2014a} run the county-level difference-in-differences  specification: 
    \begin{equation}\label{eq:tva_spillover}
        y_{c, 2000} - y_{c, 1940} = \alpha + \text{TVA}_c \tau + X_{c, 1940} \beta + (\varepsilon_{c, 2000} - \varepsilon_{c, 1940})
    \end{equation} 

    \begin{itemize}
        \item $y$ are outcomes for agricultural employment and manufacturing employment.
        \item $\text{TVA}_c$ is the treatment variable
        \item $X_{c, 1940}$ allow for different long-term trends based on covariates in 1940. 
    \end{itemize}

    They trim the sample using a logit regression to predict treatment using $X_{c,1940}$ and then keep control units in the top 75\% of predicted probability.
\end{frame}

\begin{frame}{Spillovers in the TVA Context}
    In our context, there is reason to believe spillovers can occur to nearby counties

    \begin{itemize}
        \item \textbf{Agriculture:} 
        \begin{itemize}
            \item Employees might be drawn to hire wages for new manufacturing jobs in Tennessee Valley (negative spillover on control units)
        \end{itemize}
        
        \item \textbf{Manufacturing:}
        \begin{itemize}
            \item Cheap electricity might be available to nearby counties (positive spillover on control units) 
            
            \item Manufacturing jobs that would have been created in the control units in the absence of treatment might move to the Tennessee Valley (negative spillover on control units)
        \end{itemize}
    \end{itemize}
\end{frame}

\begin{frame}{Specification including spillovers}
    \begin{equation}
        \Delta y_c = \alpha + \text{TVA}_i \tau + \sum_{d \in \text{Dist}} \text{Ring}(d)\delta_d + X_{i, 1940} \beta + \Delta \varepsilon_c
    \end{equation} 

    \begin{itemize}
        \item $\text{Ring}(d)$ is a set of indicators for being in the following distance bins (in miles) from the Tennessee Valley Authority: 
        \[ d \in \left\{ (0, 50], (50, 100], (100, 150] (150, 200]\right\} \]
    \end{itemize}
\end{frame}

\imageframe{../../figures/figure-tva-sample_slides.pdf}

\begin{frame}
\begin{table}[ht]
    \caption{Effects of Tennessee Valley Authority on Decadel Growth, 1940-2000}
    \label{tab:tva}
    \renewcommand{\arraystretch}{1.2}

    \begin{adjustbox}{width = 1\textwidth, center}
        \begin{threeparttable}
            \begin{tabular}{@{} l c@{\extracolsep{20pt}}c@{\extracolsep{4pt}}cccc @{}}
                % Head
                \toprule

                & \multicolumn{1}{c}{\textbf{Diff-in-Diff}} & \multicolumn{5}{c}{\textbf{Diff-in-Diff with Spillovers}} \\ 
                \cmidrule{2-2} \cmidrule{3-7}
                & & & TVA between & TVA between & TVA between & TVA between \\ 
                & TVA & TVA & 0-50 mi. & 50-100 mi. & 100-150 mi. & 150-200 mi. \\ 
                \textit{Dependent Var.} & (1) & (2) & (3) & (4) & (5) & (6) \\

                % Body
                \midrule
                
                Agricultural employment     & $-0.0514^{***}$ & $-0.0739^{***}$& $-0.0371^{***}$&    $-0.0164$   & $-0.0298^{***}$&  $-0.0157^{*}$ \\
                &   $(0.0114)$   &   $(0.0142)$   &   $(0.0002)$   &   $(0.0114)$   &   $(0.0096)$   &   $(0.0088)$

                \onslide<2->{
                    \\
                    Manufacturing employment    & $0.0560^{***}$ &    $0.0350$    & $-0.0203^{***}$&    $-0.0245$   &  $-0.0331^{*}$ & $-0.0296^{**}$ \\
                    &   $(0.0161)$   &   $(0.0218)$   &   $(0.0006)$   &   $(0.0282)$   &   $(0.0189)$   &   $(0.0142)$ 
                }
               
                \\ \bottomrule
            \end{tabular}
            
            % Notes 
            \begin{tablenotes}\footnotesize

                \item $^{*} p< 0.1$; $^{**} p < 0.05$; $^{***} p < 0.01$.
            \end{tablenotes}
        \end{threeparttable}
    \end{adjustbox}
\end{table}    

\end{frame}


\begin{frame}{Identification Strategies and Place-Based Policies}

    The literature on federal Enterprise Zones, place-based policy that gives tax breaks to businesses that locate within the boundary, has found conflicting results, suggesting positive or near-zero effects of the program \citep{Neumark_Young_2019}. 


\end{frame}

\begin{frame}{Identification Strategies and Place-Based Policies}
    
    \begin{itemize}
        \item \citet{Busso_Gregory_Kline_2013} compare census tracts in Empowerment Zones to census tracts that qualified and were rejected from the program. They find significant large reduction of poverty.
        
        \pause
        \item \citet{Neumark_Kolko_2010} compare census tracts in Empowerment Zones to census tracts within 1,000 feet of the Zone. They find near-zero effects on poverty.
    \end{itemize}

    \pause
    My framework can explain both of these results. If census tracts just outside the Empowerment Zones also benefit from the policy, then the estimates of \citet{Neumark_Kolko_2010} are attenuated towards zero

\end{frame}

\section{Event Study}

\begin{frame}{Gardner (2021) Overview}

    \[ 
        y_{it} = \mu_i + \mu_t + \tau D_{it} + \varepsilon_{it}
    \]

    The problem with estimating this by OLS is that the treatment variable becomes residualized $\tilde{D}_{it}$ and this leads to all sorts of problems... (see new diff-in-diff literature)

    \pause

    Gardner (2021) recommends a two-step approach:

    1. Estimate $\mu_i$ and $\mu_t$ using never-treated/not-year-treated observations ($D_{it}$ = 0). Then subtract off $\hat{\mu}_i$ and $\hat{\mu}_t$.
        
    \pause
    2. Then, regress $y_{it} - \hat{\mu}_i - \hat{\mu}_t \equiv \tilde{y}_{it}$ on $\tau D_{it}$ (or event study leads/lags). This estimate is unbiased because $D_{it}$ is non-residualized (standard errors require adjusting).
\end{frame}

\begin{frame}{Controlling for Spillovers in Staggered Treatment Timing}
    \[ 
        y_{it} = \mu_i + \mu_t + \tau D_{it} + {\color{coral} \tau_{\text{spill,control}} \text{Within}_{it} * (1-D_{it}) } + {\color{kelly} \tau_{\text{spill,treat}} \text{Within}_{it} * D_{it}} + \varepsilon_{it}
    \]

    Adjust two-step approach:

    \pause

    1. Estimate $\mu_i$ and $\mu_t$ using observations that aren't yet treated/affected by spillovers ($D_{it} = 0$ and $Within_{it} = 0$). Then subtract off $\hat{\mu}_i$ and $\hat{\mu}_t$.  
    
    2. Then, regress $\tilde{y}_{it}$ on $\tau D_{it} + {\color{coral} \tau_{\text{spill,control}} \text{Within}_{it} * (1-D_{it}) } + {\color{kelly} \tau_{\text{spill,treat}} \text{Within}_{it} * D_{it}}$ (or interacted event study leads/lags). This estimate is unbiased because $D_{it}$ is non-residualized (standard errors require adjusting).

\end{frame}

\begin{frame}{Community Health Centers}
    \begin{citecolor}\citet{Bailey_Goodman_Bacon_2015}\end{citecolor} study the creation of federal community health centers between 1965 and 1974
    
    \textbf{Research Question:} Do low-/no-cost health services lower the mortality rate of the treated counties? \emph{New Question:} Do these effects spread to neighboring counties?
\end{frame}

\begin{frame}
    \begin{figure}[tb!]
        \caption{Effects of Establishment of Community Health Centers}
        \begin{center}
            \resizebox{0.9\textwidth}{!}{
                \includegraphics[width=\textwidth]{../../figures/figure-chc-es_original.pdf}
            }
        \end{center}
    \end{figure}
\end{frame}

\begin{frame}
    \begin{figure}[tb!]
        \caption{Direct and Spillover Effects of Community Health Centers}
        \begin{center}
            \resizebox{0.9\textwidth}{!}{
                \includegraphics[width=\textwidth]{../../figures/figure-chc-es_combined.pdf}
            }
        \end{center}
    \end{figure}
\end{frame}

\section{Conclusion}

\begin{frame}{Conclusion}
    \begin{itemize}
        \item I decomposed the TWFE estimate into the direct effect and two spillover terms
        
        \item I showed that a set of concentric rings allows for estimation of the direct effect of treatment and they are able to model spillovers well
        
        \item For place-based policies, I show the importance of considering spatial spillovers when estimating treatment effects
        
        \item More generally, identification strategies that use very close control units in order to minimize differences in unobservables should consider the problems with treatment effect spillovers.
    \end{itemize}
\end{frame}





















\end{document}
