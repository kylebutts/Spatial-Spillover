\documentclass[aspectratio=169]{beamer}
% Metropolis Theme ------------------------------------------------------------------------------
\usetheme{metropolis} % Use metropolis theme


% Title ------------------------------------------------------------------------------
\title{Difference-in-Differences with Spatial Spillovers}
\date{\today}
\author{Kyle Butts}
% \institue{}

% xcolor and define colors -------------------------
\usepackage{xcolor}
% https://www.materialpalette.com/colors
\definecolor{red}{HTML}{c62828}
\definecolor{orange}{HTML}{ef6c00}
\definecolor{green}{HTML}{2e7d32}
\definecolor{blue}{HTML}{1565c0}
\definecolor{purple}{HTML}{283593}
\definecolor{maroon}{HTML}{AF3335}
\definecolor{dark-maroon}{HTML}{5D0F0D}
\definecolor{teal}{HTML}{00695c}
\definecolor{bluegrey}{HTML}{455a64}

\definecolor{buff-gold}{HTML}{CFB87C}
\definecolor{buff-grey}{HTML}{565A5C}
\definecolor{buff-lightgrey}{HTML}{A2A4A3}
\definecolor{buff-black}{HTML}{000000}


% Enumerate/Itemize --------------------------------
\usepackage{enumitem}
\setlist[itemize]{label= \textbullet}

% Hyperlinks ---------------------------------------
\usepackage{hyperref}
\hypersetup{
    colorlinks= true,
    citecolor= dark-maroon,
    linkcolor= dark-maroon,
    filecolor= dark-maroon,      
    urlcolor= dark-maroon,
}

% AMS ----------------------------------------------
\usepackage{amsmath}
\usepackage{amsfonts}

% AMS Theorem --------------------------------------
\usepackage{amsthm}

% note, theorem is the name that goes in \begin{} and Theorem is the name displayed as Theorem 1
\newtheorem{theorem}{Theorem}
\newtheorem{assumption}{Assumption}

% Font ---------------------------------------------
\usepackage[libertine]{newtxmath}


% Section and Subsection Styling -------------------
\usepackage{titlesec}

\titleformat{\section}
  {\Large \bf \color{teal}}
  {\thesection.}
  {1em}
  {}
  
\titleformat{\subsection}
  {\large \it \color{teal}}
  {\thesubsection}
  {1em}
  {}

% Better Abstract ----------------------------------
\renewenvironment{abstract}
{
  \centerline
  {\large \bfseries \color{teal} Abstract}
  \begin{quote}
}
{
  \end{quote}
}

% Margins ------------------------------------------
\usepackage[margin=1.5in]{geometry}

% Line Spacing -------------------------------------
\renewcommand{\baselinestretch}{1.5}

% Remove annoying over-full box --------------------
\vfuzz2pt 
\hfuzz2pt

% Packages for Better Tables ------------------------
\usepackage{booktabs}
\usepackage{longtable}
\usepackage{lscape}



% --------------------------------------------------

\addbibresource{references.bib}

\usepackage{adjustbox}
\usepackage{tabularx}
\usepackage{booktabs}
\usepackage{threeparttable}
\usepackage{dcolumn} 

% Table Highlighting
\usepackage[beamer,customcolors]{hf-tikz}
\usetikzlibrary{calc}
\usetikzlibrary{fit,shapes.misc}

% To set the hypothesis highlighting boxes red.
\tikzset{hl/.style={
    set fill color=red!80!black!40,
    set border color=red!80!black,
  },
}
\newcommand\marktopleft[1]{%
    \tikz[overlay,remember picture] 
        \node (marker-#1-a) at (0,1.5ex) {};%
}
\newcommand\markbottomright[1]{%
    \tikz[overlay,remember picture] 
        \node (marker-#1-b) at (0,0) {};%
    \tikz[red, ultra thick, overlay, remember picture, inner sep=4pt]
        \node[draw, rectangle, fit=(marker-#1-a.center) (marker-#1-b.center)] {};%
}


% ------------------------------------------------------------------------------
\begin{document}

% ------------------------------------------------------------------------------
\maketitle
% ------------------------------------------------------------------------------

\begin{frame}{Spatial Spillovers}
    Researchers aim to estimate the average treatment effect on the treated: 
    \[
        \tau \equiv \mathbb{E} \left[ Y_{i1}(1) - Y_{i1}(0) \ \vert \ D_{i} = 1 \right]
    \]
    
    \vspace{5mm}
    Spillover effects are when effect of treatment extend over boundaries such as states, counties, etc.
    
    \begin{itemize}
        \item e.g. large employer opening/closing in a county have positive employment effects on nearby counties
    \end{itemize}
\end{frame}


\begin{frame}{Bias from Spatial Spillovers}
    
    \onslide<1->{
        The canonical difference-in-differences estimate is: 
        \only<1>{
            \[ 
                \hat{\tau} = \underbrace{\hat{\mathbb{E}} \left[ Y_{i1} - Y_{i0} \mid D_i = 1 \right]}_{\text{Counterfactual Trend} \ + \ \tau} - 
                \underbrace{\hat{\mathbb{E}} \left[ Y_{i1} - Y_{i0} \mid D_i = 0 \right]}_{\text{Counterfactual Trend}}
            \]
        }
        \only<2>{
            \[ 
                \hat{\tau} = \underbrace{\hat{\mathbb{E}} \left[ Y_{i1} - Y_{i0} \mid D_i = 1 \right]}_{\text{Counterfactual Trend} \ + \ \tau} - 
                \underbrace{\hat{\mathbb{E}} \left[ Y_{i1} - Y_{i0} \mid D_i = 0 \right]}_{\substack{\text{Counterfactual Trend} \\[2mm] \ + \ \text{\color{purple} Spillover on Control}}}
            \]
        }
        \only<3>{
            \[ 
                \hat{\tau} = \underbrace{\hat{\mathbb{E}} \left[ Y_{i1} - Y_{i0} \mid D_i = 1 \right]}_{\substack{\text{Counterfactual Trend} \ + \ \tau \\[2mm] \ + \ \text{\color{red} Spillover on Treated}}} - 
                \underbrace{\hat{\mathbb{E}} \left[ Y_{i1} - Y_{i0} \mid D_i = 0 \right]}_{\substack{\text{Counterfactual Trend} \\[2mm] \ + \ \text{\color{purple} Spillover on Control}}}
            \]
        }
    } 

    Two problems in presence of spillover effects:
    
    
    \begin{itemize}
        \onslide<2->{
            \item {\bf \color{purple} Spillover onto Control Units:} 
            
            Nearby ``control'' units fail to estimate counterfactual trends because they are affected by treatment
        }
        
        \onslide<3->{
            \vspace{2.5mm}
            \item {\bf \color{red} Spillover onto other Treated Units:} 
            
            Treated units are also affected by nearby units and therefore combines ``direct'' effects with spillover effects
        }
    \end{itemize}

\end{frame}



% Grey out overlays
\setbeamercovered{transparent}


\begin{frame}{Contribution}

    {\footnotesize
    
    \underline{Difference-in-Differences Estimation with Spillovers} 
    
    \begin{citecolor}
        [\citet{Clarke_2017}, \citet{Berg_Streitz_2019}, and \citet{Verbitsky-Savitz_Raudenbush_2012}]
    \end{citecolor}
    
    \begin{itemize}
        \item I generalize the work by deriving the bias in terms of potential outcomes and show how to estimate treatment effects by removing spillover effects
    \end{itemize}


    \pause
    \underline{Potential Outcome Framework for Spillovers}
    
    \begin{citecolor}
        [\citet{Miguel_Kremer_2004}, \citet{Vazquez-Bare_2019}, \citet{Angelucci_DiMaro_2016}, \citet{Angrist_2014}]        
    \end{citecolor}

    \begin{itemize}
        \item I focus on a spatial setting whereas these papers consider spillovers in network clusters
    \end{itemize}


    \pause
    \underline{Place-based Policies}

    \begin{citecolor}
        [\citet{Kline_Moretti_2014b}, \citet{Kline_Moretti_2014a}, \citet{Busso_Gregory_Kline_2013}, and \citet{Neumark_Kolko_2010}]
    \end{citecolor}

    \begin{itemize}
        \item I highlight the need to control for general equilibrium effects to properly estimate local effects of policies
    \end{itemize}
    }
\end{frame}


% ------------------------------------------------------------------------------
\section{Theory}
% ------------------------------------------------------------------------------


\begin{frame}{Potential Outcomes Framework}
    For exposition, I will label units as counties. Assume all treatment occurs at the same time (2-periods or pre-post averages).\footnote{I extend this into an event study framework in the paper, but the intuition is the same as in the $2 \times 2$ setting.}
    
    \begin{itemize}
        \item $Y_{it}(D_i, \textcolor{blue}{h(\vec{D}, i)})$ is the potential outcome of county $i \in \{ 1, \dots, N \}$ at time $t$ with treatment status $D_i \in \{0, 1\}$.
        
        \pause
        \item $\vec{D} \in \{0,1\}^N$ is the vector of all units treatments.
        
        \pause
        \item The function $\textcolor{blue}{h(\vec{D}, i)}$ maps the entire treatment vector into an `exposure mapping' which can be a scalar or a vector.
        
        \pause
        \item No exposure is when $\textcolor{blue}{h(\vec{D}, i)} = \vec{0}$.
    \end{itemize}
\end{frame}

% Grey out
% \setbeamercovered{transparent}

\begin{frame}{Examples of $h_i(\vec{D})$}
    
    Examples of $h_i(\vec{D})$:
    
    \begin{itemize}
        \item \textbf{Treatment within $x$ miles:}
        
        $\textcolor{blue}{h(\vec{D}, i)} = max_j \ 1(d(i, j) \leq x)$ where $d(i,j)$ is the distance between counties $i$ and $j$. 

        \begin{itemize}
            \item e.g. library access where $x$ is the maximum distance people will travel
            
            \item Spillovers are non-additive
        \end{itemize}

    \end{itemize}
\end{frame}

\imageframe{../../figures/figure-spill_within_large.png}

\begin{frame}{Examples of $h_i(\vec{D})$}
    
    Examples of $h_i(\vec{D})$:
    
    \begin{itemize}
        \item \textbf{Treatment within $x$ miles:}
        
        $\textcolor{blue}{h(\vec{D}, i)} = max_j \ 1(d(i, j) \leq x)$ where $d(i,j)$ is the distance between counties $i$ and $j$. 

        \begin{itemize}
            \item e.g. library access where $x$ is the maximum distance people will travel
            
            \item Spillovers are non-additive
        \end{itemize}
        
        \vspace{2.5mm}
        \item \textbf{Number of Treated within $x$ miles:}
        
        $\textcolor{blue}{h(\vec{D}, i)} = \sum_{j = 1}^k 1(d(i, j) \leq x)$. 

        \begin{itemize}
            \item e.g. large factories opening
            
            \item Agglomeration economies suggest spillovers are additive
        \end{itemize}

    \end{itemize}
\end{frame}


\imageframe{../../figures/figure-spill_within_large_additive.png}


\begin{frame}{Estimand of Interest}
    Estimand of Interest: \[ 
        \textcolor{green}{\tau_{\text{direct}}} \equiv \textcolor{green}{\mathbb{E}\left[ Y_{i,1}(1, 0) - Y_{i,1}(0, 0) \mid D_i = 1\right]}
    \]

    \vspace{10mm}
    This is the direct effect in the absense of exposure to spillovers.
\end{frame}


\begin{frame}{Parallel Trends}
    I assume a modified version of the parallel counterfactual trends assumption: 
    \begin{align*}
        &\mathbb{E}\big[ \underbrace{Y_{i,1}(0, \textcolor{blue}{\vec{0}}) - Y_{i,0}(0, \textcolor{blue}{\vec{0}})}_{\text{Counterfactual Trend}} \mid D_i = 1 \big] \\
        = \ &\mathbb{E}\big[ \underbrace{Y_{i,1}(0, \textcolor{blue}{\vec{0}}) - Y_{i,0}(0, \textcolor{blue}{\vec{0}})}_{\text{Counterfactual Trend}} \mid D_i = 0 \big],
    \end{align*}

    \vspace{5mm}
    In the complete absence of treatment (not just the absence of individual $i$'s treatment), the change in potential outcomes from period 0 to 1 would not depend on treatment status
    
\end{frame}

\begin{frame}{What does Diff-in-Diff identify?}
    With the parallel trends assumption and random assignment of $D_i$, I decompose the diff-in-diff estimate as follows: 
        
    \begin{align*}
        \mathbb{E}\left[\hat{\tau}\right] &= \underbrace{\mathbb{E} \left[ Y_{i1} - Y_{i0} \mid D_i = 1 \right] - \mathbb{E} \left[ Y_{i1} - Y_{i0} \mid D_i = 0 \right]}_{\text{Difference-in-Differences}} \\[3mm]
        \pause&= 
        \textcolor{green}{\mathbb{E} \left[ Y_{i1}(1, 0) - Y_{i1}(0, 0) \mid D_i = 1 \right]} \\
        &\quad + \quad 
        \textcolor{red}{\mathbb{E} \left[ Y_{i1}(1, h_i(\vec{D})) - Y_{i1}(1, 0) \mid D_i = 1 \right]} \\ 
        &\quad - \quad  
        \textcolor{purple}{\mathbb{E} \left[ Y_{i1}(0, h_i(\vec{D})) - Y_{i1}(0, 0) \mid D_i = 0 \right]} \\[3mm]
        &= \textcolor{green}{\tau_{\text{direct}}} + \textcolor{red}{\tau_{\text{spillover, treated}}} - \textcolor{purple}{\tau_{\text{spillover, control}}}
    \end{align*}

\end{frame}

\begin{frame}{Biased Estimate for $\tau_{\text{direct}}$}
    \[ 
        \mathbb{E}[\hat{\tau}_{\text{diff-in-diff}}] - \textcolor{green}{\tau_{\text{direct}}} = \textcolor{red}{\tau_{\text{spillover, treated}}} - \textcolor{purple}{\tau_{\text{spillover, control}}}    
    \]
\end{frame}




% ------------------------------------------------------------------------------
\section{Estimation}
% ------------------------------------------------------------------------------

\begin{frame}{Problems with "Dropping Bad Controls"}
    It is common in empirical applications to drop control units near the treated units when estimating the direct effect of treatment.

    This is not recommended for two reasons:

    \begin{itemize}
        \item[1.] If not all control units that experience spillover effects are removed, then bias remains
        
        \item[2.] There is a second source of bias, the spillover effects on treated units, that still remains
    \end{itemize}

    

\end{frame}


% ------------------------------------------------------------------------------
\section{Estimation with Spillovers}
% ------------------------------------------------------------------------------

\begin{frame}{Spillovers as estimand of interest}
    Until now, we assumed our estimand of interest is $\textcolor{green}{\tau_{\text{direct}}}$.
    
    However, the two other spillover effects are of interest as well:
    \begin{itemize}
        \item $\textcolor{purple}{\tau_{\text{spillover, control}}}$: Do the benefits of a treated county come at a cost to neighbor counties? 
        
        \item $\textcolor{red}{\tau_{\text{spillover, treated}}}$: Does the estimated effect change based on others treatment? (This is what you should consider if you are a policy maker)
    \end{itemize}
    
    To estimate the spillover effects, we have to parameterize $h(\vec{D}, i)$ function and the potential outcomes function $Y_i(D_i, h(\vec{D}, i))$.
\end{frame}

\begin{frame}{Robustness to Misspecification}
    Generate data using the same data-generating process as before but with different spillover functions:

    \[ 
        y_{\it} = \mu_t + \mu_i + 2 D_{it} + \beta_{\text{spill,control}} * (1-D_{it}) h(\vec{D}, i) + \varepsilon_{it}
    \]


    Then, I estimate each data-generating process using (potentially) misspecified $\tilde{h}(\vec{D}, i)$ and report the average estimate bias.

\end{frame}

\begin{frame}{Results}
    \textbf{I find that an indicator for being Within $x$ miles from treated area will remove all bias so long as the indicator contains all the affected units.}

\end{frame}

\begin{frame}{Estimation of Spillover Effects}
    In a lot of settings, estimating the spillover effects are also an estimand of interest.

    I repeat the same exercise and estimate the spillover effects for each control unit, $\hat{\beta}_{\text{spill, control}} * \tilde{h}(\vec{D}, i)$.

    Then calculate \[ 
        1 - \frac{
                \overbrace{\sum_{i: D_i = 0} (\beta_{\text{spill, control}} h(\vec{D}, i) - \hat{\beta}_{\text{spill, control}} \tilde{h}(\vec{D}, i))^2}^{\text{Mean Square Prediction Error}}
            }{
                \underbrace{\sum_{i: D_i = 0} (\beta_{\text{spill, control}} h(\vec{D}, i))^2}_{\text{Normalization}}
            }    
    \]

    This gives the proportion of spillovers explained by $\tilde{h}(\vec{D}, i)$

\end{frame}

\imageframe{../../figures/figure-spill_ring.png}

\begin{frame}{Results}
    \textbf{Rings perform best at estimating spillover effects.}
\end{frame}






\section{Conclusion}

\begin{frame}{Conclusion}
    \begin{itemize}
        \item I decomposed the TWFE estimate into the direct effect and two spillover terms
        
        \item I showed that a set of concentric rings removes two spillover terms from treatment effect estimate and models spillovers well
        
        \item For place-based policies, I show the importance of considering spatial spillovers when estimating treatment effects
    \end{itemize}
\end{frame}



\end{document}